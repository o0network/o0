% File: diploma.tex

\documentclass[
    candidate, % document type
    subf, % use and configure subfig package for nested figure numbering
]{disser}

% Add this line to set Times font
\usepackage{mathptmx} % Use Times font for text and math

% Кодировка и язык
\usepackage[T2A]{fontenc} % поддержка кириллицы
\usepackage[utf8]{inputenc} % кодировка исходного текста
\usepackage[english,russian]{babel} % переключение языков

% Геометрия страницы и графика
\usepackage[left=3cm, right=1cm, top=2cm, bottom=2cm]{geometry} % поля страницы
\usepackage{graphicx} % подключение графики
\usepackage{pdfpages} % вставка pdf-страниц

% Таблицы
\usepackage{array} % расширенные возможности для работы с таблицами
\usepackage{tabularx} % автоматический подбор ширины столбцов
\usepackage{dcolumn} % выравнивание чисел по разделителю

% Математика
\usepackage{bm} % полужирное начертание для математических символов
\usepackage{amsmath} % дополнительные математические возможности
\usepackage{amssymb} % дополнительные математические символы

% Библиография и ссылки
\usepackage{cite} % поддержка цитирования
\usepackage{hyperref} % создание гиперссылок

% Прочее
\usepackage{color} % работа с цветом
\usepackage{epstopdf} % конвертация eps в pdf
\usepackage{multirow} % объединение ячеек таблиц по вертикали
\usepackage{afterpage} % вставка материала после текущей страницы
\usepackage[font={normal}]{caption} % настройка подписей к рисункам и таблицам
\usepackage[onehalfspacing]{setspace} % полуторный интервал
\usepackage{fancyhdr} % установка колонтитулов
\usepackage{listings} % поддержка вставки исходного кода

% Установка шрифта Times New Roman
\renewcommand{\rmdefault}{ftm}

% Настройка стиля страницы
\pagestyle{fancy}      % Использование стиля "fancy" для оформления страниц
\fancyhf{}              % Очистка текущих значений колонтитулов
\fancyfoot[C]{\thepage} % Установка номера страницы в нижнем колонтитуле по центру
\renewcommand{\headrulewidth}{0pt} % Удаление разделительной линии в верхнем колонтитуле

% Установка глубины оглавления
\setcounter{tocdepth}{2}

\begin{document}

% Включение титульного листа (первая страница файла Title.pdf)
\includepdf[pages={1}]{assets/TitlePage.pdf}

% Аннотация, Содержание, Введение
\renewcommand{\contentsname}{\centerline{\large СОДЕРЖАНИЕ}}
\tableofcontents

\newpage
\begin{center}
  \textbf{\large ВВЕДЕНИЕ}
\end{center}
\addcontentsline{toc}{chapter}{ВВЕДЕНИЕ}

% Глава 1
\newpage
\begin{center}
  \textbf{\large 1. Аналитический обзор }
\end{center}
\refstepcounter{chapter}
\addcontentsline{toc}{chapter}{1. Аналитический обзор }

?? \cite{nakamoto2008bitcoin} ???

% Глава 2
\newpage
\begin{center}
  \textbf{\large 2. Разработка и исследование приложения }
\end{center}
\refstepcounter{chapter}
\addcontentsline{toc}{chapter}{2. Разработка и исследование приложения }


% Заключение
\newpage
\begin{center}
  \textbf{\large ЗАКЛЮЧЕНИЕ}
\end{center}
\refstepcounter{chapter}
\addcontentsline{toc}{chapter}{ЗАКЛЮЧЕНИЕ}

% Библиографический список
\newpage
\addcontentsline{toc}{chapter}{СПИСОК ИСПОЛЬЗОВАННЫХ ИСТОЧНИКОВ} % это будет отображаться в содержании
\renewcommand{\bibsection}{\centering\textbf{\large СПИСОК ИСПОЛЬЗОВАННЫХ ИСТОЧНИКОВ}} % смена названия библиографии по умолчанию
\bibliographystyle{bibformat}
\bibliography{bibliography}

% Приложение
\newpage
\begin{center}
  \textbf{\large ПРИЛОЖЕНИЕ А}
\end{center}
\refstepcounter{chapter}
\addcontentsline{toc}{chapter}{ПРИЛОЖЕНИЕ А}

\end{document}